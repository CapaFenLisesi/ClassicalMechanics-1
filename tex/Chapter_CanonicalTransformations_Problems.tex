\section{Problems}

\begin{problem}{Invariance of Poisson Brackets} Show that the Poisson Bracket of two functions, $U$, $V$, is invariant even under time-dependent canonical transformations, $Q_i(q_i,p_i,t)$, $P_i(q_i,p_i,t)$:
\begin{align*}
\{U,V\}_{q,p}=\{U,V\}_{Q,P}
\end{align*}
\label{prob_CT_1}
\end{problem}

\begin{problem}{Canonical prescription and angular momentum}
Use the canonical prescription for quantization to show that the angular momentum in the z-direction is quantized.
\label{prob_CT_2}
\end{problem}

\begin{problem}{Canonical transformation types}
Show that the transformation equations for type 3 and type 4 canonical transformations are given by the following relations, respectively:
\begin{align*}
F&=F_3(p_i,Q_i,t)+\sum_ip_iq_i \\
q_i&=-\die{F_3}{p_i}\\
P_i&=-\die{F_3}{Q_i}\\
\end{align*}
and:
\begin{align*}
F&=F_4(p_i,P_i,t)+\sum_ip_iq_i-\sum_iQ_iP_i\\
q_i&=-\die{F_4}{p_i}\\
Q_i&=\die{F_4}{P_i}\\
\end{align*}
\label{prob_CT_3}
\end{problem}

\begin{problem}{Identifying canonical transformations}
Determine and show which of the following transformations are canonical:\\
\textbf{a)}:
\begin{align*}
Q&=\frac{1}{2}(q^2+p^2)\\
P&=-\tan^{-1}(\frac{q}{p})
\end{align*}
\textbf{b)}:
\begin{align*}
Q&=\sqrt{2q}e^t \cos(p)\\
P&=\sqrt{2q}e^{-t}\sin(p)
\end{align*}
\textbf{c)}:
\begin{align*}
Q&=\ln\frac{\sin(p)}{q}\\
P&=q\cot(p)
\end{align*}
\label{prob_CT_4}
\end{problem}

\begin{problem}{Canonical transformation of a Hamiltonian}
\textbf{a)} Show that the following transformation is canonical:
\begin{align*}
Q&=\frac{1}{2}(q^2+p^2)\\
P&=-\tan^{-1}(\frac{q}{p})
\end{align*}
\textbf{b)} If the Hamiltonian, $H(q,p)$, is given by:
\begin{align*}
H=\frac{1}{2}(q^2+p^2)
\end{align*}
Write an expression of the new Hamiltonian, $K(Q,P)$, and write the equations of motion for the new canonical variables ($Q$, $P$).\\\\
\textbf{c)} Show that the transformed Hamiltonian, $K(Q,P)$, is a constant of the motion\\
\textbf{d)} Write an expression for the Lagrangian of this system in terms of $q$ and $\dot q$\\
\textbf{e)} Write out the equation of motion for $\ddot q$, and give an example of a physical system that is described by this Lagrangian/Hamiltonian
\label{prob_CT_5}
\end{problem}