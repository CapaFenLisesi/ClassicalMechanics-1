%Copyright 2016 R.D. Martin
%This book is free software: you can redistribute it and/or modify it under the terms of the GNU General Public License as published by the Free Software Foundation, either version 3 of the License, or (at your option) any later version.
%
%This book is distributed in the hope that it will be useful, but WITHOUT ANY WARRANTY; without even the implied warranty of MERCHANTABILITY or FITNESS FOR A PARTICULAR PURPOSE.  See the GNU General Public License for more details, http://www.gnu.org/licenses/.
\section{Problems}
\begin{problem}{Spherical pendulum} Write the Lagrangian for a spherical pendulum (a bob of mass $m$ attached to a mass-less rigid rod of length $l$) using spherical coordinates and identify all conserved quantities. A spherical pendulum is the generalized case of the simple pendulum when the mass is not constrained to swing in a plane. 
\label{prob_ConsSym_1}
\end{problem}

\begin{problem}{Two masses on a spring} Two masses, $m_1$ and $m_2$, are connected by a spring of rest length $l$ and spring constant $k$. The two masses are constrained to move in one dimension, along the x-axis, on a friction-less surface.\\
\capfig{0.3\textwidth}{figures/ConnectBlocks.png}{\label{fig:ConnectBlocks} Two blocks connected by a spring slide on a friction-less surface. (Problem \ref{prob_ConsSym_2})}
\textbf{a)} Give the Lagrangian for the system and write the equations of motion for the two masses\\
\textbf{b)} Show that the total linear momentum in the x-direction, $P_x$, is conserved ($P_x=m_1v_1+mv_2$)\\
\textbf{c)} List all conserved quantities for the system.
\label{prob_ConsSym_2}
\end{problem}

\begin{problem}{Arbitrary potential}Calculate the conserved quantities for the Lagrangian:
\begin{align*}
L=\frac{1}{2}m_1\dot{q}_1^2+\frac{1}{2}m_2\dot{q}_2^2-V(aq_1-bq_2)
\end{align*}
where $a$ and $b$ are constants, and V() is some unknown function of the linear combination $aq_1-bq_2$.
\label{prob_ConsSym_3}
\end{problem}

\begin{problem}{Spring pendulum with two masses} The pendulum in figure \ref{fig:SpringPendulumTwoMass} is constructed with two masses, $m_1$ and $m_2$, a spring of constant $k$ and rest length $d$. Mass $m_2$ is fixed at the end of a mass-less rigid rod of length $l$, while $m_1$ can slide without friction along the rod and is connected to the spring.\\
\capfig{0.2\textwidth}{figures/SpringPendulumTwoMass.png}{\label{fig:SpringPendulumTwoMass} A pendulum with two masses, one of which is connected to the pivot point by a spring (Problem \ref{prob_ConsSym_4}).}
\textbf{a)} Give the Lagrangian for the system and write the equations of motion for the two masses\\
\textbf{b)} List all conserved quantities for the system.
\label{prob_ConsSym_4}
\end{problem}


%\begin{problem}{General rotation}
%Calculate the $f_i$ coefficients for a general infinitesimal rotation about an arbitrary axis in cartesian coordinates given by a vector, $\vec{a}$, with components $a_x$, $a_y$, $a_z$.
%\label{prob_ConsSym_4}
%\end{problem}
%-Bead on a helix wire

