\section{Problems}

\begin{problem}{The Atwood Machine}
Figure \ref{fig:Atwood_Lag} shows three point masses, $m_1$, $m_2$, and $m_3$ that are part of an Atwood machine with two pulleys. The pulleys are both solid disks of mass $M$ and radius $R$. The ropes slide slide without slipping on the pulleys. The top rope has a total length $L_1$ and the bottom rope has a total length $L_2$.
\\
\textbf{a)}Choose suitable generalized coordinates and write the Lagrangian for the system in terms of the generalized coordinates\\
\textbf{b)}Use the Lagrangian to obtain the equations of motion for the generalized coordinates\\
\capfig{0.5\textwidth}{figures/Atwood.png}{\label{fig:Atwood_Lag} An Atwood machine.}
\label{prob_Lagrange_1}
\end{problem}

\begin{problem}{Simple pendulum}
The pendulum in Figure \ref{fig:SimplePendulum_Lag} is composed of a mass $m$ attached to a mass-less rigid rod of the length $L$. The pendulum can swing in the xy-plane. 
\capfig{0.2\textwidth}{figures/SimplePendulum.png}{\label{fig:SimplePendulum_Lag}The mass $m$ is attached by mass-less rigid rod of length $L$ and free to swing in the xy-plane under the action of gravity. (Problem \ref{prob_Lagrange_2})}\\
\textbf{a)}Choose suitable generalized coordinates and write the Lagrangian for the system in terms of the generalized coordinates\\
\textbf{b)}Use the Lagrangian to obtain the equations of motion for the generalized coordinates\\
\textbf{c)}Repeat parts a) and b) to obtain the equations of motion for the case where the rod has a mass $M$\\
\label{prob_Lagrange_2}
\end{problem}

\begin{problem}{Moving pendulum}
The pendulum in Figure \ref{fig:MovingPendulum_Lag} is composed of a mass $m$ attached to a mass-less rigid rod of the length $L$. The pendulum can swing in the xy-plane. The pivot point of the bar moves downwards at a fixed, known, speed $v$.
\capfig{0.15\textwidth}{figures/MovingPendulum.png}{\label{fig:MovingPendulum_Lag}The mass $m$ is attached by mass-less rigid rod of length $L$ and free to swing in the xy-plane under the action of gravity. The pivot point moves with a fixed, known velocity, $v$, and was at the origin at time $t=0$. (Problem \ref{prob_Lagrange_3})}\\
\textbf{a)}Choose suitable generalized coordinates and write the Lagrangian for the system in terms of the generalized coordinates\\
\textbf{b)}Use the Lagrangian to obtain the equations of motion for the generalized coordinates.
\label{prob_Lagrange_3}
\end{problem}


\begin{problem}{Two masses and two springs}
\label{prob_Lagrange_4}
The figure shows two masses, $m_1$ and $m_2$, each connected to two springs with spring constants $k_1$ and $k_2$. Mass $m_1$ is constrained to slide without friction along the x-axis, whereas mass $m_2$ is constrained to move in the vertical direction, constrained by a massless frictionless vertical rod that is attached to $m_1$. Both springs have a resting length of $L$.
\capfig{0.2\textwidth}{figures/TwoMassesTwoSprings.png}{Two masses and two springs, problem \ref{prob_Lagrange_4}}\\
\textbf{a)}Choose suitable generalized coordinates and write the Lagrangian for the system in terms of the generalized coordinates\\
\textbf{b)}Use the Lagrangian to obtain the equations of motion for the generalized coordinates.
\end{problem}

\begin{problem}{Pendulum with a spring}
\label{prob_Lagrange_5}
The figure shows a bead of mass, $m$, that can slide freely along a long massless rail which has one end fixed at the origin, forming a pendulum. The mass is connected to the pivot point at the origin by a massless spring of resting length, $L$, and spring constant $k$. The motion is constrained to be in the vertical plane (gravity pointing downwards in the figure).
\capfig{0.2\textwidth}{figures/SpringPendulum.png}{Pendulum with a spring, problem \ref{prob_Lagrange_5}}\\
\textbf{a)}Choose suitable generalized coordinates and write the Lagrangian for the system in terms of the generalized coordinates\\
\textbf{b)}Use the Lagrangian to obtain the equations of motion for the generalized coordinates.
\end{problem}


\begin{problem}{Compound pendulum}
\label{prob_Lagrange_6}
The figure shows two beads of mass, $m_1$ and $m_2$, that form a compound pendulum. Mass $m_1$ is connected by a massless rigid rod of length $L_1$ to a fixed pivot point at the origin. Mass $m_@$ is connected to mass $m_1$ by a massless rigid rod of length $L_2$. The motion is constrained to be in the vertical plane
\capfig{0.35\textwidth}{figures/CompoundPenduluum2.png}{Pendulum with a spring (Problem \ref{prob_Lagrange_6}).}\\
\textbf{a)}Choose suitable generalized coordinates and write the Lagrangian for the system in terms of the generalized coordinates\\
\textbf{b)}Use the Lagrangian to obtain the equations of motion for the generalized coordinates.
\end{problem}

\begin{problem}{Block on a hemisphere}Find the Lagrangian for a block of mass $m$ sliding under the influence of gravity without friction along a hemisphere of radius R (Figure \ref{fig:BlockOnHemisphere}). Use the method of Lagrange multipliers to determine the normal force exerted by the hemisphere on the block. Find the point at which the block will fall off the hemisphere if it started at rest at the top.
\capfig{0.2\textwidth}{figures/BlockOnHemisphere.png}{\label{fig:BlockOnHemisphere} A block sliding without friction on a hemisphere (Problem \ref{prob_Lagrange_7})}
\label{prob_Lagrange_7}
\end{problem}

\begin{problem}{Sphere on a hemisphere}Find the Lagrangian for a sphere of mass $m$ and radius $r$ that is rolling without slipping along a hemisphere of radius $R$ (Figure \ref{fig:BallOnHemisphere}). Use the method of Lagrange multipliers to determine the forces exerted by the hemisphere on the sphere. Find the point at which the sphere will fall off the hemisphere if it started at rest at the top.
\capfig{0.2\textwidth}{figures/BallOnHemisphere.png}{\label{fig:BallOnHemisphere}A ball rolling without slipping on a hemisphere (Problem \ref{prob_Lagrange_8}).}
\label{prob_Lagrange_8}
\end{problem}

\begin{problem}{Sliding blocks}
The block of mass $m$ in Figure \ref{fig:SlidingBlocks} can slide without friction on the wedge of mass $M$ that itself can slide without friction on the ground. The dimensions of the small block can be assumed to be negligible, the dimensions of the wedge are shown in the figure, and the motion is constrained to be in the xy-plane.
\capfig{0.3\textwidth}{figures/SlidingBlocks.png}{\label{fig:SlidingBlocks}The mass $m$ can slide without friction on the wedge of mass $M$ which itself can slide with no friction on the ground (Problem \ref{prob_Lagrange_9})}\\
\textbf{a)}Choose suitable generalized coordinates and write the Lagrangian for the system in terms of the generalized coordinates\\
\textbf{b)}Use the Lagrangian to obtain the equations of motion for the generalized coordinates.
\label{prob_Lagrange_9}
\end{problem}

\begin{problem}{Sphere on a wedge}
The sphere of mass $m$ and radius $r$ in Figure \ref{fig:BallOnWedge} can roll without slipping on the wedge of mass $M$ that itself can slide without friction on the ground. The dimensions of the wedge are shown in the figure, and the motion is constrained to be in the xy-plane.
\capfig{0.3\textwidth}{figures/BallOnWedge.png}{\label{fig:BallOnWedge}The sphere of mass $m$ and radius $r$ rolls without slipping on the wedge of mass $M$ which itself can slide with no friction on the ground (Problem \ref{prob_Lagrange_10})}\\
\textbf{a)}Choose suitable generalized coordinates and write the Lagrangian for the system in terms of the generalized coordinates\\
\textbf{b)}Use the Lagrangian to obtain the equations of motion for the generalized coordinates.
\label{prob_Lagrange_10}
\end{problem}

\begin{problem}{Sliding blocks with a spring}
The block of mass $m$ in Figure \ref{fig:SlidingBlocksSpring} can slide without friction on the wedge of mass $M$ that itself can slide without friction on the ground. The dimensions of the small block can be assumed to be negligible, the dimensions of the wedge are shown in the figure, and the motion is constrained to be in the xy-plane. The small block is attached to a spring with rest length $d$ and spring constant $k$.
\capfig{0.3\textwidth}{figures/SlidingBlocksSpring.png}{\label{fig:SlidingBlocksSpring}The mass $m$ can slide without friction on the wedge of mass $M$ which itself can slide with no friction on the ground. The mass $m$ is connected to $M$ with a spring with spring constant $k$ and rest length $d$ (Problem \ref{prob_Lagrange_11})}\\
\textbf{a)}Choose suitable generalized coordinates and write the Lagrangian for the system in terms of the generalized coordinates\\
\textbf{b)}Use the Lagrangian to obtain the equations of motion for the generalized coordinates.
\label{prob_Lagrange_11}
\end{problem}

\begin{problem}{Routhian and a spherical pendulum}
A spherical pendulum is constructed by using a mass $m$ and a mass-less rod of length $L$ as seen in in Figure \ref{fig:SphericalPendulum} .
\capfig{0.2\textwidth}{figures/SphericalPendulum.png}{\label{fig:SphericalPendulum}A spherical pendulum with mass $m$ attached to a rigid mass-less rod of length $L$ (Problem \ref{prob_Lagrange_12})}\\
\textbf{a)}Write out the Lagrangian for the spherical pendulum using $\theta$ and $\phi$ as the generalized coordinates\\
\textbf{b)}Write out expressions for any conserved quantities.\\
\textbf{c)}Show that the system is described by the following Routhian:
\begin{align*}
R = \frac{1}{2}mL^2\dot\theta^2-\frac{1}{2}\frac{\beta_\phi^2}{mL^2\sin^2\theta}+mgL\cos\theta
\end{align*}
\\
\label{prob_Lagrange_12}
\end{problem}


%\begin{problem}{Straight line in polar coordinates}
%Show that the following equations of motion in cylindrical coordinates lead to motion in a straight line when projected on the horizontal plane:
%The equations of motion from Lagrange's equation are:
%\begin{align*}
%\ddot{r}&=r\dot{\theta}^2\\
%\ddot{\theta}&=0\\
%\ddot{z}&=-mg
%\end{align*}
%\label{prob_Lagrange_X}
%\end{problem}